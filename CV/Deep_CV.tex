% LaTeX file for resume 
% This file uses the resume document class (res.cls)

\documentclass[margin]{res} 
\usepackage{graphicx}
\usepackage{latexsym} 
\usepackage{fancyhdr}
\usepackage{wrapfig}
\usepackage{textcomp}
\usepackage{amsmath}
\usepackage{biblatex}
\addbibresource{biblio.bib}
% the margin option causes section titles to appear to the left of body text 
\textwidth=5.2in % increase textwidth to get smaller right margin
%\usepackage{helvetica} % uses helvetica postscript font (download helvetica.sty)
%\usepackage{newcent}   % uses new century schoolbook postscript font 

\begin{document} 
 
\moveleft.5\hoffset\centerline{\large\bf Curriculum Vitae}
\moveleft\hoffset\vbox{\hrule width\resumewidth height 1pt}\smallskip 

 
\begin{resume} 
 
\section{Name}
Deep Ray

%\section{Date of Birth}
%1st July, 1989

\section{Affiliation}
Computational Mathematics and Simulation Science (MCSS)  \\ \'{E}cole Polytechnique F\'{e}d\'{e}rale de Lausanne \\ CH-1015 Lausanne, Switzerland  \\
E-mail: deep.ray@epfl.ch\\
Webpage: deepray.github.io
 
\section{Employment}
{\it Postdoctoral researcher}, MCSS, EPFL
 \begin{itemize} \itemsep -2pt  % reduce space between items
 \item 
July, 2017 to present
\item Involved in the development of machine learning strategies to resolve existing bottle-necks in numerical methods.
\end{itemize}

\section{Education}
{\it Ph.D. in Mathematics}, Tata Institute of Fundamental Research (TIFR-CAM)
 \begin{itemize} \itemsep -2pt  % reduce space between items
 \item September, 2013 to May, 2017
 \item Advisors: Dr. Praveen Chandrashekar (TIFR-CAM) and Prof. Siddhartha Mishra (ETH Zurich and adjunct faculty TIFR-CAM)
  \item Developed and analysed a high-order entropy stable parallelized finite volume solver for the compressible Euler and Navier-Stokes equations on unstructured meshes. 
  \item Awarded the Harish Chandra Memorial Award for the best Ph.D. thesis.
      
 \end{itemize}

 {\it M.Phil in Mathematics}, TIFR-CAM
 \begin{itemize} \itemsep -2pt  % reduce space between items
 \item July, 2012 to September, 2013
 \item Advisor: Dr. Praveen Chandrashekar
 \item Worked on numerical schemes for the Euler and Navier-Stokes Equations that preserve entropy, kinetic-energy and vorticity, and performed multi-dimensional simulations for the same.
 \end{itemize}

 {\it M.Sc in Mathematics}, TIFR-CAM
 \begin{itemize} \itemsep -2pt  % reduce space between items
 \item August, 2010 to May, 2012
 %\item Cumalative score: 82.37 $\%$
 \end{itemize}
 
 {\it B.Sc (Honours) in Mathematics}, Hindu College, University of Delhi
 \begin{itemize} \itemsep -2pt  % reduce space between items
 \item July, 2007 to June, 2010
% \item Cumalative score: 83.89 $\%$
 \end{itemize}
 
% {\it All India Senior School Certificate Examination}, Delhi Public School, R.K. Puram, Delhi
% \begin{itemize} \itemsep -2pt  % reduce space between items
% \item Date: May, 2007
% \item Score: 93.00 $\%$
% \end{itemize} 
% 
% {\it All India Secondary School Certificate Examination}, Delhi Public School, R.K. Puram, Delhi
% \begin{itemize} \itemsep -2pt  % reduce space between items
% \item Date: May, 2005
% \item Score: 89.20 $\%$
% \end{itemize}  

\section{Research Visits}
 {\it Visiting Research Student}, Seminar for Applied Mathematics, ETH Zurich
 \begin{itemize} \itemsep -2pt  % reduce space between items
 \item May to October, 2014 and August 2015 to May 2016            
 \item Visited Prof. Siddhartha Mishra to work with him and his research group.
 \end{itemize}

 \section{Publications}

\textbf{Publication in Journals}                
            \begin{itemize}            
   
           \item {\it An artificial neural network as a troubled-cell indicator.}\\
            D. Ray and J. S. Hesthaven.\\
            Journal of Computational Physics, Vol. 367 (15), pp 166-191 (2018).

            \item {\it An entropy stable finite volume scheme for the two dimensional Navier-��Stokes equations on triangular grids.}\\
            D. Ray and P. Chandrashekar.\\
             Applied Mathematics and Computation, Vol. 314, pp. 257-286 (2017).

             \item {\it Convergence of fully discrete schemes for diffusive-dispersive conservation laws with discontinuous flux.}\\
              U. Koley, R, Dutta and D. Ray. \\
              ESAIM: Mathematical Modelling and Numerical Analysis, Vol. 50(5), pp.1289-1331, (2016).
              
              \item {\it Entropy stable schemes on two-dimensional unstructured grids for Euler equations.}\\
              D. Ray, P. Chandrashekar, U. S. Fjordholm, S. Mishra. \\
              Communications in Computational Physics, Vol. 19(5), pp. 1111-1140, (2016).
            
              \item {\it A sign preserving WENO reconstruction method.}\\
              U. S. Fjordholm, D. Ray. \\
              Journal of Scientific Computing, Vol. 68(1), pp. 42-63, (2015).
             
              \item {\it Entropy stable schemes for compressible Euler equations.}\\
              D. Ray and P. Chandrashekar.\\
              Int. J. Numer. Anal. Model. Ser. B, no. 4, p. 335-352 (2013).
             \end{itemize}



\textbf{Publication in Conference Proceedings}                
            \begin{itemize}           
              \item {\it A Third-Order Entropy Stable Scheme for the Compressible Euler Equations.}\\
              D. Ray.\\
              Theory, Numerics and Applications of Hyperbolic Problems II. HYP 2016. Springer Proceedings in Mathematics and Statistics, vol 237., 2018

              \item {\it Kinetic energy preserving and entropy stable finite volume schemes for compressible Euler and Navier-Stokes equations.}\\
              D. Ray and P. Chandrashekar.\\
              14th Annual CFD Symposium - Aeronautical Society of India, IISc, Bangalore, 10-11 August, 2012.
             \end{itemize}

\textbf{Submitted for publication}                
            \begin{itemize}           
              
              \item {\it Detecting troubled-cells on two-dimensional unstructured grids using a neural network}. (submitted, 2018) \\
               D. Ray and J. S. Hesthaven.
               
              \item {\it Non-intrusive reduced order modelling of unsteady flows using artificial neural networks with application to a combustion problem.} (submitted, 2018)\\
              Q. Wang, J. S. Hesthaven and D.Ray.
           
              \item {\it Multi-level Monte Carlo finite difference methods for fractional conservation laws with random data.} (submitted, 2018)\\
              U. Koley, D. Ray and T. Sarkar.

             \end{itemize}

             
\section{Selected Talks}

              {\it An artificial neural network as a troubled-cell indicator} (10th July, 2018)\\
              SIAM Annual Meeting 2018, Portland
              
              {\it An artificial neural network for detecting discontinuities} (11th March, 2018)\\
              7th International Conference on High Performance Scientific Computing, Hanoi 
              
              {\it A high-resolution energy preserving method for the rotating shallow water equation} (27th September, 2017)\\
               European Conference on Numerical Mathematics and Advanced Applications (ENUMATH-2017), Voss
              
              {\it A third order entropy stable scheme for the compressible Euler equations} (4th August, 2016)\\
               XVI International Conference on Hyperbolic Problems (HYP2016), Aachen 
              
%              Talk: {\it A sign preserving WENO reconstruction} (23rd November, 2015)\\
%               Department of Mathematics, University of W$\ddot{\text{u}}$rzburg 
              
              {\it A sign preserving WENO reconstruction} (14th August, 2015)\\
              International Conference on Industrial and Applied Mathematics, Beijing
              
%              Talk: {\it A sign preserving WENO reconstruction} (11th June, 2015)\\
%              Department of Applied Mathematics, University of Washington, Seattle
              
              {\it Entropy stable schemes for compressible flows on unstructured meshes} (20th December, 2014)\\
               Conference on Computational PDEs, Finite Element Meet, TIFR-CAM  
              
              {\it Entropy stable schemes for compressible flows on unstructured meshes} (9th November, 2014)\\
               The 5th International Conference on Scientific Computing and Partial Differential Equations, HKBU, Hong Kong              
             
%              Poster: {\it Entropy stable schemes for compressible flows on unstructured meshes} (9th September, 2014)\\
%               Workshop on the Analysis and Numerical Approximation of PDEs, ETH Zurich
%              
%              Talk: {\it Entropy stable schemes for compressible flows} (9th July, 2014)\\
%               Department of Mathematics, University of W$\ddot{\text{u}}$rzburg             

%\section{Conferences \\ and Workshops} 
%             \begin{itemize}
%             \item {\it XVI International Conference on Hyperbolic Problems}, RWTH Aachen  (1st-5th August, 2016)
%             \item {\it Academic Industry Modelling Week}, University of Zurich  (9th-13th November, 2015) 
%             
%              \item {\it International Conference on Industrial and Applied Mathematics}, Beijing, China  (10th-14th August, 2015) 
%              
%               \item {\it Conference on Computational PDEs, Finite Element Meet}, TIFR-CAM (18th - 20th December, 2014) 
%               
%               \item{\it The 5th International Conference on Scientific Computing and Partial Differential Equations}, HKBU, Hong Kong (8th - 12th December, 2014)
%               
%               \item{\it Workshop on the Analysis and Numerical Approximation of PDEs}, ETH Zurich (8th - 10th September, 2014)
%               
%               \item{\it CIME-CIRM Workshop on Mathematical Models and Methods for Living Systems}, Levico Terme, Italy (1st - 5th September, 2014)
%
%               \item {\it Swiss Numerics Day 2014}, University of Zurich (25th April, 2014)
%               
%   
%               \item{\it Workshop on Optimization with PDE constraints}, TIFR-CAM (25th November - 6th December, 2013)
%   
%               
%               \item{\it Compact course on Discontinuous Galerkin method for time-dependent convection-dominant PDEs, by Prof. Chi-Wang Shu }, TIFR-CAM (4th - 5th July, 2013)
%               
%               
%               \item{\it International Conference on Conservation Laws and Applications}, TIFR-CAM (1st - 3rd July, 2013)
%               
%               \item{\it IFCAM Summer School on Numerics and Control of PDEs}, IISc, Bangalore (22nd July - 2nd August, 2013)
%               
%               \item{\it CIMPA Summer Research School on Current Trends in Computational Methods for PDEs}, IISc, Bangalore (24th June - 19th July, 2013)
%               
%               \item{\it Workshop on Theoretical and Computational Aspects of Nonlinear Waves}, \\NPDE-TCA, IIT-Bombay (27th - 31st May, 2013)
%
%               \item{\it Advanced Workshop on Non-Standard Finite Element Methods}, NPDE-TCA, \\IIT Bombay (11th - 15th Febraury, 2013)
%
%               \item{\it Heterogeneous Parallel Programming}, University of Illinois at Urbana-Champaign's \\ Online Coursera offering (28th November, 2012 - 28th January, 2013)
%               
%               \item{\it 14th Annual CFD Symposium}, CFD Division - Aeronautical Society of India, IISc (10th - 11th August, 2012)
%
%               \item{\it Instructional Workshop on FEM}, TIFR-CAM (2nd - 13th July, 2012)
%               
%               \item{\it Data Assimilation Research Program}, TIFR-CAM (4th - 23rd July, 2011)
%
%               \item{\it Visiting Students’ Research Programme}, TIFR Mumbai (15th June - 10th July, 2009)
%               \end{itemize}          
            
              
\section{Teaching Experience} 
   
               \begin{itemize} \itemsep -2pt
               \item Teaching Assistant for graduate course on Numerical Methods for Conservation Laws, at EPFL (September - December, 2018)
                \item Teaching Assistant for graduate course on Numerical Methods for Conservation Laws, at EPFL (September - December, 2017)
               \item Teaching Assistant for graduate course on Computational Partial Differential Equations, at TIFR-CAM (January - May, 2015)
               \item Teaching Assistant for graduate course on Numerical Analysis, at TIFR-CAM  (August - December, 2013)
               \item Organised numerical sessions for optimal control at the IFCAM Summer School on Numerics and Control of PDEs-2013, at the Indian Institute of Science, Bangalore. 
%               \begin{itemize}
%                 \item Participants were given a crash course on MATLAB coding and ODE-solvers.
%                 \item The models considered were the inverted pendulum,  Burgers equation the heat equation in both 1D and 2D set-up.
%                 \item Numerical evaluation of feedback control and solving the estimation problem for noisy partial observations were discussed and implemented.
%               \end{itemize}
               \item Teaching Assistant for graduate course on Numerical Analysis, at TIFR-CAM   (August - December, 2012)
               \end{itemize}              

%\section{Additional Activities} 
%   
%               \begin{itemize} \itemsep -2pt
%               \item Member of the Students Seminar Series (S$^3$) committee (August 2012 - December 2013). The purpose of this committee was to organise and oversee talks by motivated students, on mathematical or other science oriented topics.
%               \item Participation with Souvik Roy in {\it Join the spirit: Find me if you can}, a competition organized by EADS.
%	      \begin{itemize}
%	          \item Cleared the first phase, which required the developed a code for human detection. The code uses the theory of locally normalized Histograms of Gradients (HOG), proposed by N. Dalal and W. J. Triggs. The code is also capable of capturing moving people in videos. The basic algorithm is available in the C opencv library.
%	          \item Currently working on the second and final phase, which requires teams to suggest an algorithm to compare two images containing human figures and assess whether both images contain the same person, within acceptable confidence levels. 
%	      \end{itemize}    
%               \item Founding member of {\it Science Forum}, a society created in 2009 at Hindu College, University of Delhi. The mandate of this society was to bring together the students from various science departments, so that they could share the knowledge gathered in their respective fields and gain insights into other associated areas.
%               \item Founding member of {\it Caucus}, a society created in 2008 at Hindu College, University of Delhi. The society served as a forum for discussing various burning national and global issues, and training students for the Model United Nations Conferences.
%               \end{itemize}                    
               
               

%\section{Extracurricular \\ Activities} 

%               {\bf Social and Organisational Skills}
%               \begin{itemize} \itemsep -2pt
%               \item During my first undergraduate year, I teamed up with other motivated students to set up {\it Caucus}, a forum for %discussing various burning national and global issues, and training students for the Model United Nations Conferences.
%               \item Along with other members of Caucus, I launched the {\it International Hindu Model United Nations}, and organized %its first conference in August 2008, with participation by students from within India and abroad. 
%               \item I received {\it Higher Commendation} as the delegate of Mexico at BITS Model United Nations 2010, organised by Bits Pilani.
%               \item In my second undergraduate year, I was elected {\it General Secretary} of the college mathematics society.
%               \item In my final undergraduate year, I was involved in setting up the {\it Science Forum}. The mandate of this society was to bring together the students from various science departments, so that they could share the knowledge gathered in their respective fields and gain insights into other associated areas. 
%               \end{itemize}

%               {\bf Artistic Skills}
%               \begin{itemize} \itemsep -2pt
%               \item I learnt the acoustic guitar for two years, after which I switched over to the western classical guitar. I have received a distinction in my Grade 2 Examination conducted by Trinity College of London. 
%               \item I participated in several inter school clay-modeling competitions and received awards, and a certificate of recognition from my school.
%               \end{itemize}

% Tabulate Computer Skills; p{3in} defines paragraph 3 inches wide
\section{Computer \\ Skills}
   \begin{tabular}{l p{3in}}
    {\bf Languages:} & C++, Fortran, Python \\

    {\bf Programming Software:} &  MATLAB \\
    
    {\bf Visualisation Software:} & Tecplot, Paraview, Gnuplot, VisIt, Gmsh \\
    
    {\bf Machine-Learning Software:} & TensorFlow
 \end{tabular}


\end{resume} 
\end{document} 



